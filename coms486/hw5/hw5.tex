\documentclass[11pt]{article}
\usepackage{mathtools}
\usepackage{mdframed}
\usepackage{fullpage}
\usepackage{amsfonts}
\usepackage{tikz}
\usepackage{fancyhdr}
\usepackage{lastpage}
\usepackage{enumitem}
\usepackage{graphicx}


%edit this for each class
\newcommand\name{John Collin Vincent}
\newcommand\classname{Com S 486}
\newcommand\assignment{Homework 5}


\newcounter{excounter}
\setcounter{excounter}{1}
\newcommand\ques[2]{\vskip 1em  \noindent\textbf{\arabic{excounter}\addtocounter{excounter}{1}.} \emph{#1} \noindent#2}
\newenvironment{question}{\ques{}\begin{quote}}{\end{quote}}
\newenvironment{subquestion}[1]{#1) \begin{quote}}{\end{quote}}

\pagestyle{fancy}
\rfoot{\name, page \thepage/\pageref{LastPage}}
\cfoot{}
\rhead{}
\lhead{}
\renewcommand{\headrulewidth}{0pt}
\renewcommand{\footrulewidth}{0pt}
\DeclarePairedDelimiter\ceil{\lceil}{\rceil}
\DeclarePairedDelimiter\floor{\lfloor}{\rfloor}


\begin{document}


    {\bf \classname \hspace{1cm} \assignment\hfill \name}
    \vskip 2em


    \begin{question}
        A and B will detect the collision at time 200. A will then start rebroadcasting at time $200 + Backoff_A$, A will then take
        $ 512 * 8 + 64 = 4160 $ bit time to fully transmit its message. B will detect that the channel is busy after its backoff time 
        and will have to wait for A to finish. B will then start at $ 4360 + Backoff_A + Backoff_B$
    \end{question}

    \begin{question}
        Since B will only start broadcasting if it thinks the channel is idle the worst case scenario for when B will start is time $t=199$
        the first bit of B will arrive at A at time $t=399$. A will detect a collision up to time $t=576$ so in the case that B collides with
        A, A will always detect the collision.
    \end{question}

    \begin{question}
        Instead of making 3 tables I'm just going to list all of the entries and specify the event that 
        causes the entry to be added after the switch receives the packet\\
        \begin{tabular}{ r | c | c | c }
                 added after  & \multicolumn{3}{ | c }{Switch Table}\\
                        & MAC & Interface & Timestamp\\\hline
                $i$     & $A$ & $I_A$ & $i + D_{prop}(A)$\\
                $ii$    & $C$ & $I_C$ & $ii + D_{prop}(C)$\\
                $iii$   & $B$ & $I_B$ & $iii + D_{prop}(B)$ 
        \end{tabular}
    \end{question}

    \begin{question}
        The first thing the computer will do is use DHCP to obtain an IP address by constructing a message in an 
        Ethernet frame and setting it to the broadcast address 255.255.255.255. Your computer will get back
        a IP address from the DHCP server and a lease time that the IP is valid for, as well as the first hop 
        address for the network, and network sub-net mask. It will then use ARP to get the hardware address
        for the first hop router and DNS server. Then the IP of the website will be resolved through DNS.
        After that a TCP connection will be established to the web server. The TCP connection will be used
        to facilitate the HTTP request and response from the server.
    \end{question}

    \begin{question}
        signal 1: (1, -1, 1, -1, 1, -1, 1, -1)\\
        signal 2: (-1, 1, -1, 1, -1, 1, -1, 1)
    \end{question}

    \begin{question}
        the bit time for just sending the payload will be $512 * 8 + 64$\\
        the time for sending the RTS, CTS, and ACK will be be 3 times the minimum packet size $3 * (512 + 64)$\\
        $time = 3*\text{SIFS} + \text{DIFS} + 512 * 8 + 3 * 512 + 4*64$\\
        $time = 3*\text{SIFS} + \text{DIFS} + 5888$
    \end{question}

\end{document}
