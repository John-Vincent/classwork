\documentclass[11pt]{article}
\usepackage{mathtools}
\usepackage{mdframed}
\usepackage{fullpage}
\usepackage{amsfonts}
\usepackage{tikz}
\usepackage{setspace}
\usepackage{fancyhdr}
\usepackage{lastpage}

\doublespacing
%edit this for each class
\newcommand\name{John Collin Vincent}
\newcommand\classname{SE329}
\newcommand\assignment{Homework 2}


\newcounter{excounter}
\setcounter{excounter}{1}
\newcommand\ques[2]{\vskip 1em  \noindent\textbf{\arabic{excounter}\addtocounter{excounter}{1}.} \emph{#1} \noindent#2}
\newenvironment{question}{\ques{}\begin{quote}}{\end{quote}}
\newenvironment{subquestion}[1]{#1) \begin{quote}}{\end{quote}}

\pagestyle{fancy}
\rfoot{\name, page \thepage/\pageref{LastPage}}
\cfoot{}
\rhead{}
\lhead{}
\renewcommand{\headrulewidth}{0pt}
\renewcommand{\footrulewidth}{0pt}
\DeclarePairedDelimiter\ceil{\lceil}{\rceil}
\DeclarePairedDelimiter\floor{\lfloor}{\rfloor}


\begin{document}


  {\bf \classname \hspace{1cm} \assignment\hfill \name}
  \vskip 2em


  Having a code of Ethics ensures that everyone understands what needs to be done in order to protect the rights of people their work affects. Following the code
  of ethics is how you know that your work will be improving the society/community that you work in instead of degrading it. The code of ethics helps remind
  people of the basic things they need to ensure their work does in order to protect their consumers.
  Following the code of ethics also protects you from legal trouble since laws are made to disincentives unethical behavior.

  Personally when I am confronted with an ethical decision, I like to evaluate my situation and consider all the possible options. It is not always possible to
  act 100\% ethically but there are always more ethical options in any situation. A specific example would be when I got a take home exam for a class. I started reading the
  questions and answering assuming that it was an open note exam. After answering the first question i saw that it was suppose to be a closed note take home exam. This
  surprised me because that doesn't really make sense, as anyone in the class could use any resource in taking the exam and there would be no way to enforce the
  closed note requirement. My decision was to stop taking/looking at the exam, review my notes, then come back and take the rest of the exam without the use of notes.
  This is a situation were there is no 100\% ethical way out I had already answered the first question using notes. But in order to be honest and responsible I decided
  that I should take the rest of the exam following the rules asked by the teacher knowing that I would get a lower score than a majority of the class that used both
  notes and online resources to take the exam. The decision that I made was done by looking at all the possible options, finish taking the exam right away without review
  and do poorly because of being unprepared, finish taking the exam with my notes and lie about following the rules, or review for the exam then finish. The last option
  seemed to be as ethical as the first option but would result in a better outcome so its the option I went with.

  My group discussed the VW scandal. This scandal was were the company VW installed a defeat device into their diesel cars to artificially lower their emission's during
  testing. This decision was made by high level executives at the company in order to make their diesel cars marketable in most western countries which have a maximum
  emission level for cars. The result was higher levels of NO$_x$ gas throughout the world. On top of using the defeat device to trick regulators, VW also paid
  scientist to conduct studies that would prove the NO$_x$ wasn't dangerous to health, which is empirically false. This NO$_x$ causes smog to form as well as, acid rain,
  lung diseases in children, and other environmental concerns. VW did develop NO$_x$ traps in their cars but they were expensive and needed filters to be replaced, so
  they would only engage these traps during testing. During this time VW was struggling financially and a feeling of desperation caused the high executives to make
  unethical decisions to benefit the bottom line.

  If you compare these actions to the IEEE code of ethics 7.8 for developers you can see that there are multiple codes
  being violated by these developers. The first code which is, "to hold paramount the safety, health, and welfare of the public..." was clearly violated because
  this device directly causes 40x the amount of NO$_x$ pollution by VW cars. The third code is, "to be honest and realistic in stating claims or estimates based on available data;",
  this device was clearly made to be dishonest in the emission claims. This device also breaks the 5th code which is to increase public understanding of tech, this device
  was clearly designed to obscure the capabilities of diesel cars. the 7th rule is also broken because they were not seeking honest feedback on their cars
  performance, and correcting errors with emissions, but rather hiding performance and making this error in emissions marketable. the 9th code is to avoid injuring
  others, and this device causes injury to large ecosystems. What the developers should have done to avoid breaking these ethical guidelines was to refuse to develop the defeat
  device for VW but rather offer to develop a system that could actually reduce the emissions for the cars.

  Once VW was caught, instead of coming clean and getting rid of the defeat device, they updated the software to be harder to detect. This breaks the 10th rule because
  now the knowledge of the defeat device was wide spread throughout the company and the people in the company didn't work for professional development and push
  colleagues to conduct themselves ethically. They instead doubled down and colluded to cheat even harder. In the end the only reason VW came clean was because the
  EPA threatened to ban VW cars from being sold in 2016, so in the end it was a cost benefit analysis that caused them to own up to their actions and not an ethical
  consideration. This device is an example of Badware because it is a device that takes control of the product away from the user. The user in under the impression
  that they are doing a certain amount of pollution when driving the car, but this device actually disables a filtration system that ended up causing the user
  to have an impact on the environment that they most likely would not be OK with doing.

  My group agreed that making the defeat device was an ethical violation on the part of the developers, and that they should have outright refused to make the device.
  Someone in the group pointed out that if they refused to make the device they would likely loose their jobs and be replaced with people the would make the device,
  so maybe they should make the device but make it in a way that it doesn't actually disable the emission control outside of the test environment like it was suppose to.
  This way the public would be better off and the company wouldn't be able to develop the device with other less ethical developers. The issue with this is that you would
  still be acting unethical by being dishonest to your employer so the only truly ethical option would be to refuse to make it and probably loose your job. our answers
  were similar in that we had the ethical consideration of the end consumer/population in mind first, and the second idea might honestly have the better result
  for the public since if you refuse to develop the device the company will still probably get the device developed by someone. The ideas were different because
  the second still had some unethical behavior, in the end we agreed that the first option was probably the best option for the developers.

  When looking at the Virtue of Ethics paper and applying it to the events of the VW scandal it seems clear to me that the Virtue that was broken the most was
  Honesty. It was found in emails between executives in the congressional investigation that the executives where openly talking about how they were lying
  and working tirelessly to be cover up their fraud. They were clearly being intentionally dishonest with the world. The second virtue that was broken would be
  Fidelity. Through trying to trick the agencies that had caught their defeat device they were being unfaithful to their promise to right their wrongs. The third
  virtue would be Integrity because they were clearly being intentionally unethical to try and get an economic upper hand. Many of the executives were knowingly
  doing things that were wrong and this is clear because they made coordinated efforts to cover up their unethical actions. The fourth virtue would be Responsibility
  because they were being irresponsible to the world population by poisoning the environment at rates that had been outlawed. I think courage is a virtue
  that is not included in the list that applies to this case because there were so many people involved in this scandal and none of them had to courage
  to speak out against the wrongs the company was committing until they were in legal trouble.


\end{document}
