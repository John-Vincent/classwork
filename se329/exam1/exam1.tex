\documentclass[11pt]{article}
\usepackage{mathtools}
\usepackage{mdframed}
\usepackage{fullpage}
\usepackage{amsfonts}
\usepackage{tikz}
\usepackage{fancyhdr}
\usepackage{lastpage}


%edit this for each class
\newcommand\name{John Collin Vincent}
\newcommand\classname{SE329}
\newcommand\assignment{exam1  Secret Key = Holland}


\newcounter{excounter}
\setcounter{excounter}{1}
\newcommand\ques[2]{\vskip 1em  \noindent\textbf{\arabic{excounter}\addtocounter{excounter}{1}.} \emph{#1} \noindent#2}
\newenvironment{question}{\ques{}\begin{quote}}{\end{quote}}
\newenvironment{subquestion}[1]{#1) \begin{quote}}{\end{quote}}

\pagestyle{fancy}
\rfoot{\name, page \thepage/\pageref{LastPage}}
\cfoot{}
\rhead{}
\lhead{}
\renewcommand{\headrulewidth}{0pt}
\renewcommand{\footrulewidth}{0pt}
\DeclarePairedDelimiter\ceil{\lceil}{\rceil}
\DeclarePairedDelimiter\floor{\lfloor}{\rfloor}


\begin{document}


  {\bf \classname \hspace{1cm} \assignment\hfill \name}
  \vskip 2em


  \begin{question}
    The resources allocated to the project affect the amount of man hours you can spend
    on the project as well as the components you can use to create the project, which in the
    end will affect what can be accomplished. If you are given a smaller budget than you
    requested you might have to use cheaper hardware for the project or use a smaller team.
    The complexity of the project affects how long you will have to work on developing
    the project to get a correct solution. If you are working on a project that has some
    complicated machine learning component to it then you will have to spend the
    development time and the resources to pay the developers in order to get a solution that
    works as specified.

  \end{question}

  \begin{question}
    They used the Rolling wave planning which leaves later task vague, this means that
    as they approach the later task in the project they were less planned out than the initial task, causing them to be behind the initial schedule. They need to put the time into
    fully developing the schedule then use schedule control techniques like adjusting leads,
    and lags.
  \end{question}
  \clearpage
  \begin{question}
    \hspace*{-.4cm}
    \begin{tabular}{|c|l|l|}
      \hline\\
      PM  & Justification & Response\\\hline


      Yes &
        this might affect the number of houses & survey the houses in the intended area for  \\ & our system will be compatible with &
        water delivery system, and adjust project
        work\\&& with most common systems.\\\hline


      No  & services like that are normally another teams & trust the service team\\ & responsibility, and the development
      team &\\ &needs to focus on creating the product & \\\hline


      Yes & user data should always be protected & run penetration test on the system, \\& &to make sure the data is secure\\\hline

      No  & That is the billing teams job & trust the accountants that do the companies billing.\\\hline

      Yes & If that is a piece that we are & do a stress test on this component,\\ & providing we need to test that it can stand & and potentially replace it will a better component\\& up
      to the conditions it will be put in \\\hline


      Yes & our equipment should be able to & stress test
      critical equipment\\ & withstand the stresses of normal use\\\hline

      Yes & that's always a concern & check specifications for the \\&& project and make sure
      all the\\&& goals are being met\\\hline

      Yes & we should design a project that actually & do test on consumers
      \\ & fills a need of the consumer & where they get to use the product \\&& and give
      feed back.\\\hline

      Yes & developing software to standards ensure ethical & include
      experience developers on the team\\& and easy to maintain
      software & and set up an environment that enforces standards\\\hline

      Yes & The product must be thoroughly tested & delay the delivery if necessary\\\hline
    \end{tabular}
  \end{question}

  \begin{question}
    I would try to program the project to a set library or environment that i knew
    would be implemented on any architecture that could be used in the future. For example
    the project components that risk having different architecture could be done in java
    because there is almost a guarantee that the JVM will be implemented for that architecture.
  \end{question}

  \begin{question}
    \begin{verbatim}
                      Camera System
        Camera           billing          follow-up
   identify speeders    send bill           email
     record license    collect payments     send email
                                           notify police
    \end{verbatim}
  \end{question}

  \clearpage
  \begin{question}
    \begin{tabular}{|c|}
      \hline\\
      Schedule\\\hline
      create camera\\\hline
      create lincense identifying \\software\\\hline
      create server endpoint for \\cameras to send data to\\\hline
      create billing system will banks api\\\hline
      create system that gets address \\from license plate and sends bill\\\hline
      create system that monitors \\payments and schedules follow ups\\\hline



    \end{tabular}
  \end{question}

\end{document}
