\documentclass[11pt]{article}
\usepackage{mathtools}
\usepackage{mdframed}
\usepackage{fullpage}
\usepackage{amsfonts}
\usepackage{tikz}            % if you delete this, you will have trouble with the included picture. You can delete the picture.



%edit this for each class
\newcommand\name{John Vincent}
\newcommand\classname{COMS 321}
\newcommand\assignment{Homework 2}



\newcounter{excounter}
\setcounter{excounter}{1}
\newcommand\question[2]{\vskip 1em  \noindent\textbf{\arabic{excounter}\addtocounter{excounter}{1}.} \emph{#1} \noindent#2}


% You can also erase this if you do not have package hancyhdr
% Fancy footnote.........
\usepackage{fancyhdr}  %% If it does not work with your latex installation, you may just delete this...
\pagestyle{fancy}
\usepackage{lastpage}
\rfoot{\name, page \thepage/\pageref{LastPage}}
\cfoot{}
\rhead{}
\lhead{}
\renewcommand{\headrulewidth}{0pt}
\renewcommand{\footrulewidth}{0pt}
\DeclarePairedDelimiter\ceil{\lceil}{\rceil}
\DeclarePairedDelimiter\floor{\lfloor}{\rfloor}



\begin{document}


  {\bf \classname \hspace{1cm} \assignment\hfill \name}
  \vskip 2em


  \question{}\\
  \begin{center}
    \begin{tabular}{r | c | c | c | c}
                           &  ALU & load & branch & total\\ \hline
                    cycles &    1 &    2 &      4 & \\ \hline
      A number in millions &   30 &   75 &     45 & 150\\ \hline
      B number in millions &   30 &   60 &     30 & 120\\ \hline
            A\% occurences &   20 &   50 &     30 & 100\\ \hline
            B\% occurences &   25 &   50 &     25 & 100\\ \hline
                     A CPI &      &      &        & 2.4\\ \hline
                     B CPI &      &      &        & 2.25\\ \hline
    \end{tabular}
  \end{center}
  \begin{center}\underline{\% Occurences}\end{center}\vspace{-.5cm}
  \begin{align*}
  total_A &= ALU_A + load_A + branch_A\\
      150 &= 30 + 75 + 45\\
        1 &= .2 + .5 + .3\\
      100 &= 20 + 50 + 30\\\\
  total_B &= ALU_B + load_B + branch_B\\
      120 &= 30 + 75 + 45\\
        1 &= .25 + .5 + .25\\
      100 &= 25 + 50 + 25
  \end{align*}\\
  \begin{center}\underline{CPI}\end{center}
  \[
    \text{CPI}     = \text{Cycles}_{\text{ALU}}(\text{ALU}\%) + \text{Cycles}_{\text{load}}(\text{load}\%) +
        \text{Cycles}_{\text{branch}}(\text{branch}\%)
  \]
  \begin{align*}
    \text{CPI}_{A} &= 1(.2) + 2(.5) + 4(.3)\\
                   &= .2 + 1 + 1.2\\
                   &= 2.4\\\\
    \text{CPI}_{B} &= 1(.25) + 2(.5) + 4(.25)\\
                   &= .25 + 1 + 1\\
                   &= 2.25\\\\
  \end{align*}
  \begin{center}\underline{Execution Time}\end{center}\vspace{-.5cm}
  \begin{align*}
    \text{E.T.} &= \text{CPI} * \text{Instructions} * \frac{\text{Time}}{\text{Cycle}}\\\\
    \text{E.T.}_{A} &= 2.4 * 150,000,000 * \frac{\text{Time}}{\text{Cycle}}_{A}\\
                    &= 360,000,000 * \frac{\text{Time}}{\text{Cycle}}_{A}\\\\
    \text{E.T.}_{B} &= 2.25 * 120,000,000 * \frac{\text{Time}}{1.2 * \text{Cycle}}_{A}\\
                    &= \frac{270,000,000}{1.2} * \frac{\text{Time}}{\text{Cycle}}_{A}\\
                    &= 225,000,000 * \frac{\text{Time}}{\text{Cycle}}_{A}\\\\
    \text{E.T.}_{B} &< \text{E.T.}_{A}
  \end{align*}

  \question{}
  \begin{center}
    \begin{tabular}{r | c | c | c | c}
                           &  ALU & load & branch & total\\ \hline
                    cycles &    2 &    4 &      6 & \\ \hline
      A number in millions &   20 &   40 &     30 & 90\\ \hline
      B number in millions &   25 &   50 &     20 & 95\\ \hline
            A\% occurences &   $22.\overline{2}$ &   $44.\overline{4}$ &     $33.\overline{3}$ & 100\\ \hline
            B\% occurences &   26 &   53 &     21 & 100\\ \hline
                     A CPI &      &      &        & $4.\overline{2}$\\ \hline
                     B CPI &      &      &        & 3.9\\ \hline
    \end{tabular}
  \end{center}
  \begin{center}\underline{\% Occurences}\end{center}\vspace{-.5cm}
  \begin{align*}
  total_A &= ALU_A + load_A + branch_A\\
       90 &= 20 + 40 + 30\\
        1 &= .\overline{2} + .\overline{4} + .\overline{3}\\
      100 &= 22.\overline{2} + 44.\overline{4} + 33.\overline{3}\\\\
  total_B &= ALU_B + load_B + branch_B\\
       95 &= 25 + 50 + 20\\
        1 &= .26 + .53 + .21\\
      100 &= 25 + 50 + 25
  \end{align*}\\
  \begin{center}\underline{CPI}\end{center}
  \[
    \text{CPI}     = \text{Cycles}_{\text{ALU}}(\text{ALU}\%) + \text{Cycles}_{\text{load}}(\text{load}\%) +
        \text{Cycles}_{\text{branch}}(\text{branch}\%)
  \]
  \begin{align*}
    \text{CPI}_{A} &= 2(.\overline{2}) + 4(.\overline{4}) + 6(.\overline{3})\\
                   &= .\overline{4} + 1.\overline{7} + 1.\overline{9}\\
                   &= 4.\overline{2}\\\\
    \text{CPI}_{B} &= 2(.26) + 4(.53) + 6(.21)\\
                   &= .52 + 2.12 + 1.26\\
                   &= 3.9\\\\
  \end{align*}
  \begin{center}\underline{Execution Time}\end{center}\vspace{-.5cm}
  \begin{align*}
    \text{E.T.} &= \text{CPI} * \text{Instructions} * \frac{\text{Time}}{\text{Cycle}}\\\\
    \text{E.T.}_{A} &= 4.\overline{2} * 90,000,000 * \frac{\text{Time}}{1.15 * \text{Cycle}}_{B}\\
                    &= \frac{380,000,000}{1.15} * \frac{\text{Time}}{\text{Cycle}}_{B}\\
                    &= 330,434,793 * \frac{\text{Time}}{\text{Cycle}}_{B}\\
    \text{E.T.}_{B} &= 3.9 * 95,000,000 * \frac{\text{Time}}{\text{Cycle}}_{B}\\
                    &= 370,500,000 * \frac{\text{Time}}{\text{Cycle}}_{B}\\\\
    \text{E.T.}_{A} &< \text{E.T.}_{B}
  \end{align*}

  \question{}\\
  \indent a)\\
  \begin{align*}
    \text{CPI} &= \frac{\text{price}_A * \text{quantity}_A + \text{price}_B * \text{quantity}_B + \text{price}_C * \text{quantity}_C}{\text{Quantity}_{\text{total}}}\\
               &= \frac{100 * 4 + 200 * 4 + 300 * 7}{15}\\
               &= 220
  \end{align*}
  \indent b)\\
  \begin{align*}
    \text{CPI} &= 100 * \frac{3}{12} + 200 * \frac{3}{12} + 300 * \frac{6}{12}\\
               &= 225
  \end{align*}
\end{document}
