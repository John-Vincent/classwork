\documentclass[11pt]{article}
\usepackage{mathtools}
\usepackage{mdframed}
\usepackage{fullpage}
\usepackage{amsfonts}
\usepackage{tikz}            % if you delete this, you will have trouble with the included picture. You can delete the picture.



%edit this for each class
\newcommand\name{John Vincent}
\newcommand\classname{COMS 321}
\newcommand\assignment{Homework 2}



\newcounter{excounter}
\setcounter{excounter}{1}
\newcommand\question[2]{\vskip 1em  \noindent\textbf{\arabic{excounter}\addtocounter{excounter}{1}.} \emph{#1} \noindent#2}


% You can also erase this if you do not have package hancyhdr
% Fancy footnote.........
\usepackage{fancyhdr}  %% If it does not work with your latex installation, you may just delete this...
\pagestyle{fancy}
\usepackage{lastpage}
\rfoot{\name, page \thepage/\pageref{LastPage}}
\cfoot{}
\rhead{}
\lhead{}
\renewcommand{\headrulewidth}{0pt}
\renewcommand{\footrulewidth}{0pt}
\DeclarePairedDelimiter\ceil{\lceil}{\rceil}
\DeclarePairedDelimiter\floor{\lfloor}{\rfloor}



\begin{document}


  {\bf \classname \hspace{1cm} \assignment\hfill \name}
  \vskip 2em


  \question{}\\
  \begin{center}
    \begin{tabular}{r | c | c | c | c}
                           &  ALU & load & branch & total\\ \hline
                    cycles &    1 &    2 &      4 & NA\\ \hline
      A number in millions &   30 &   75 &     45 & 150\\ \hline
      B number in millions &   30 &   60 &     30 & 120\\ \hline
            A\% occurences &   20 &   50 &     30 & 100\\ \hline
            B\% occurences &   25 &   50 &     25 & 100\\ \hline
                     A CPI &   NA &   NA &     NA & 2.4\\ \hline
                     B CPI &   NA &   NA &     NA & 2.25\\ \hline
    \end{tabular}
  \end{center}
  \begin{center}\underline{\% Occurences}\end{center}\vspace{-.5cm}
  \begin{align*}
  total_A &= ALU_A + load_A + branch_A\\
      150 &= 30 + 75 + 45\\
        1 &= .2 + .5 + .3\\
      100 &= 20 + 50 + 30\\\\
  total_B &= ALU_B + load_B + branch_B\\
      120 &= 30 + 75 + 45\\
        1 &= .25 + .5 + .25\\
      100 &= 25 + 50 + 25
  \end{align*}\\
  \begin{center}\underline{CPI}\end{center}\vspace{-.5cm}
  \begin{align*}
    \text{CPI}     &= \text{Cycles}_{\text{ALU}}(\text{ALU}\%) + \text{Cycles}_{\text{load}}(\text{load}\%) +
        \text{Cycles}_{\text{branch}}(\text{branch}\%)\\\\
    \text{CPI}_{A} &= 1(.2) + 2(.5) + 4(.3)\\
                   &= .2 + 1 + 1.2\\
                   &= 2.4\\\\
    \text{CPI}_{B} &= 1(.25) + 2(.5) + 4(.25)\\
                   &= .25 + 1 + 1\\
                   &= 2.25\\\\
  \end{align*}
\end{document}
