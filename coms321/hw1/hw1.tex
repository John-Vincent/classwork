\documentclass[11pt]{article}
\usepackage{mathtools}
\usepackage{mdframed}
\usepackage{fullpage}
\usepackage{amsfonts}
\usepackage{marginnote}
\usepackage{tikz}            % if you delete this, you will have trouble with the included picture. You can delete the picture.




\newcommand\name{John Vincent} %%%%%%%%%%%%%%  WRITE YOUR NAME HERE



\newcounter{excounter}
\setcounter{excounter}{1}
\newcommand\question[2]{\vskip 1em  \noindent\textbf{\arabic{excounter}\addtocounter{excounter}{1})} \emph{#1} \noindent#2}


% You can also erase this if you do not have package hancyhdr
% Fancy footnote.........
\usepackage{fancyhdr}  %% If it does not work with your latex installation, you may just delete this...
\pagestyle{fancy}
\usepackage{lastpage}
\rfoot{\name, page \thepage/\pageref{LastPage}}
\cfoot{}
\rhead{}
\lhead{}
\renewcommand{\headrulewidth}{0pt}
\renewcommand{\footrulewidth}{0pt}
\DeclarePairedDelimiter\ceil{\lceil}{\rceil}
\DeclarePairedDelimiter\floor{\lfloor}{\rfloor}



\begin{document}


  {\bf COMS 321  \hspace{1cm} HW 1 due 8/31\hfill \name}
  \vskip 2em


  \question{}
    \begin{align*}
      \frac{B_{time}}{A_{time}} &= 1 + \frac{n}{100}\\
      \frac{40}{20} &= 1 + \frac{n}{100}\\
      100 &= n
    \end{align*}
    \indent b is the correct answer

  \question{}
    \begin{align*}
      \text{speedup} &= \frac{1}{(1-f_1-f_2...-f_n)+\frac{f_1}{s_1}...+\frac{f_n}{s_n}}\\
                     &= \frac{1}{(1-.35) + \frac{.35}{15}}\\
                     &= 1.48
    \end{align*}

  \question{}\\
    \indent a)
      \begin{align*}
        \text{speedup} &= \frac{1}{1-.8 + \frac{.8}{20}}\\
                       &= 4.16
      \end{align*}
    \indent b)
      \begin{align*}
        \text{speedup} &= \frac{1}{1-.2 + \frac{.2}{80}}\\
                       &= 1.24
      \end{align*}
    \indent c)
      \begin{align*}
        \text{speedup} &= \frac{1}{1-.9 + \frac{.9}{10}}\\
                       &= 5.26
      \end{align*}
    \indent d)
      \begin{align*}
        \text{speedup} &= \frac{1}{1-.1 + \frac{.1}{90}}\\
                       &= 1.1
      \end{align*}
    \indent the best option for improving the overall speedup of the program
    is option c.

  \question{}\\
    \begin{equation*}
      \text{module speedup} = 1 + \frac{\text{percent speedup of module}}{100}
    \end{equation*}
    \indent a)
      \begin{align*}
        \text{speedup} &= \frac{1}{1-.8 + \frac{.8}{1 + .2}}\\
                       &= 1.15
      \end{align*}
    \indent b)
      \begin{align*}
        \text{speedup} &= \frac{1}{1-.2 + \frac{.2}{1 + .8}}\\
                       &= 1.09
      \end{align*}
    \indent c)
      \begin{align*}
        \text{speedup} &= \frac{1}{1-.9 + \frac{.9}{1 + .1}}\\
                       &= 1.08
      \end{align*}
    \indent d)
      \begin{align*}
        \text{speedup} &= \frac{1}{1-.1 + \frac{.1}{1 + .9}}\\
                       &= 1.04
      \end{align*}
    \indent a is the best option for overall speedup.

  \question{}
    \begin{quote}
      a)
      \begin{align*}
        \text{speedup} &= \frac{1}{1-.6 + \frac{.6}{10}}\\
                       &= 2.17
      \end{align*}
      b)
      \begin{align*}
        \text{speedup} &= \frac{1}{1-.75 + \frac{.75}{10}}\\
                       &= 3.07
      \end{align*}
      c)
      \begin{align*}
        2.75 &= \frac{1}{1-x + \frac{x}{10}}\\
        2.75(1-x + \frac{x}{10}) &= 1\\
        2.75(10-9x) &= 10\\
        x &= -\frac{10}{2.75 * 9} + \frac{10}{9} \\
        x &= .707
      \end{align*}
      70.7 percent of the program needs to use the floating point processor
    \end{quote}

  \question{}
    \begin{quote}
      a)
      \begin{align*}
        \text{speedup} &= \frac{1}{1-.6 + \frac{.6}{1.3}}\\
                       &= 1.16
      \end{align*}
      b)
      \begin{align*}
        \text{cost/speedup 1} &= 50000(1-.7 + \frac{.7}{1.3})\\
                              &= 41923\\
        \text{cost/speedup 2} & 53000(1-.5 + \frac{.6}{2})\\
                              &= 39750
      \end{align*}
      I would select the second option because the cost per percent speed increase is lower.
    \end{quote}

\end{document}
