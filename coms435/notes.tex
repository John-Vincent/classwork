\documentclass[11pt]{article}
\usepackage{mathtools}
\usepackage{mdframed}
\usepackage{fullpage}
\usepackage{amsfonts}
\usepackage{tikz}
\usepackage{fancyhdr}
\usepackage{lastpage}
\usepackage{enumitem}
\usepackage{changepage}


%edit this for each class
\newcommand\name{John Collin Vincent}
\newcommand\classname{Com S 435}
\newcommand\assignment{Notes}


\newcounter{excounter}
\setcounter{excounter}{1}
\newcommand\ques[2]{\vskip 1em  \noindent\textbf{\arabic{excounter}\addtocounter{excounter}{1}.} \emph{#1} \noindent#2}
\newenvironment{question}{\ques{}\begin{quote}}{\end{quote}}
\newenvironment{subquestion}[1]{#1) \begin{quote}}{\end{quote}}

\newcommand\tab{\hspace*{1cm}}

\pagestyle{fancy}
\rfoot{\name, page \thepage/\pageref{LastPage}}
\cfoot{}
\rhead{}
\lhead{}
\renewcommand{\headrulewidth}{0pt}
\renewcommand{\footrulewidth}{0pt}
\DeclarePairedDelimiter\ceil{\lceil}{\rceil}
\DeclarePairedDelimiter\floor{\lfloor}{\rfloor}


\begin{document}


  {\bf \classname \hspace{1cm} \assignment\hfill \name}
  \vskip 2em


  Book\\
  Mining of Massiv Data Sets -- Rajaraman \& Ullman \& Leskovic\\

  Date: 8/21/18\\\\

  most algorithms are based on three probablisitc hypotheticals
  \begin{enumerate}[label=\alph*),leftmargin=2cm]
    \item toss a coin x times
    \item roll a dice x times
    \item pick a random element from set x
  \end{enumerate}

  the sample space is the set of possible outcomes the sample point is a specific
  observed outcome.

  any subset of the sample space $\Omega$ is called an Event E

  $E\subseteq\Omega$

  


\end{document}
