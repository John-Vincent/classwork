\documentclass[11pt]{article}
\usepackage{mathtools}
\usepackage{mdframed}
\usepackage{fullpage}
\usepackage{amsfonts}
\usepackage{tikz}
\usepackage{graphicx}
\usepackage{fancyhdr}
\usepackage{lastpage}


%edit this for each class
\newcommand\name{John Collin Vincent}
\newcommand\classname{Com S}
\newcommand\assignment{Homework 10}


\newcounter{excounter}
\setcounter{excounter}{1}
\newcommand\ques[2]{\vskip 1em
\noindent\textbf{\arabic{excounter}\addtocounter{excounter}{1}.} \emph{#1} \noindent#2}
\newenvironment{question}{\ques{}\begin{quote}}{\end{quote}}
\newenvironment{subquestion}[1]{#1\begin{quote}}{\end{quote}}


\pagestyle{fancy}
\rfoot{\name, page \thepage/\pageref{LastPage}}
\cfoot{}
\rhead{}
\lhead{}
\renewcommand{\headrulewidth}{0pt}
\renewcommand{\footrulewidth}{0pt}
\DeclarePairedDelimiter\ceil{\lceil}{\rceil}
\DeclarePairedDelimiter\floor{\lfloor}{\rfloor}


\begin{document}


  {\bf \classname \hspace{1cm} \assignment\hfill \name}
  \vskip 2em


  \begin{question}
    \begin{subquestion}{Union}
      Assume there is a Turing Machine $M_1$ that accepts the language $L_1$ with the
      alphabet $\Sigma_1=\{a_1,a_2\ldots a_n\}$, and some Turing Machine $M_2$ that accpets the
      language $L_2$ with the alphabet $\Sigma_2=\{b_1, b_2\ldots b_m\}$. By the Turing thesis
      we know that both $M_1$ and $M_2$ can be represented by a single tape single head
      machine. We can then construct $M_1\cup M_2$ by making a two tape two head machine
      $M_3$ where the first tape is $M_1$ and the second tape is
      $M_2$ and $\Sigma_3 = \Sigma_1 \cup \Sigma_2$. The set of states and
      the set oftransitions will also be the union of the states
      and transitions in $M_1$ and $M_2$. For any input $w$ you can run the two heads
      on $w$ and if either of them reach their halting state then $w$ will be accepted.
      By the Turing thesis we know that this multi tape Turing Machine can be converted into
      a single tape single head Machine.\\

      I would build a machine that would copy the input $w$ and split into two tapes with one
      head each. one head would use the transitions from $M_1$ and one head would use the
      transitions from $M_2$, if either of the heads halt then $w$ is accecpted.
    \end{subquestion}

    \begin{subquestion}{Intersection}
      since Turing acceptable languages are proven to be closed under union, demorgans law
      proves that they must also be closed under intersection
      $\overline{\overline{M_1}\cup\overline{M_2}}= M_1 \cap M_2$\\

      In order to create this Machine I would use two tapes each with a copy of input $w$, this
      could be done by starting with rewriting $w$. Then each head would use the transitions
      from one of the two languages, The Machine then only accepts if both heads halt.
    \end{subquestion}

    \begin{subquestion}{Reversal}
      Give a Turing Machine $M$ that can recognize the language $L$ a Machine $M_r$ can be
      constructed that will recongnize $L^{\reflectbox{R}}$. Given input $w\in
      L^{\reflectbox{R}}$ on the tape
      $M_r$ could just move to the other side of $w$ which would essentially
      create $w^{\reflectbox{R}}$ then from there $M_r$ can follow the same process as $M$ to
      recognize the language since $w^{\reflectbox{R}}\in L$.
    \end{subquestion}
  \end{question}

  \begin{question}
    if you have a Turing Machine $M_L$ that can accept a language $L$ and a Turing Machine
    $M_K$ that can accept a language $K$ then you can construct a machine $M_{LK}$ that
    accpets $K$ concatenated onto $L$. This Turing machine will have some input $w$ that
    can be split into two partitions $xy$ where $x\in L, y\in K$, this can be done
    non determinantly so that a copy of the TM can be made that partitions the input at every
    symbol and if any of them halt then the input is an instance of $LK$
  \end{question}

  \begin{question}
    \begin{subquestion}{a)}
      \begin{subquestion}{transition function:}
        $\delta: K \times\Sigma\times\Gamma_1\times\Gamma_2\rightarrow(K\cup \{h\}) \times
        (\Sigma\cup\{L,R\})$
      \end{subquestion}
      \begin{subquestion}{configuration:}
        $(Q\cup{h})\times\Sigma^*\times\Sigma\times(\Sigma^*(\Sigma\backslash\{\#\})\cup
        \epsilon)\times\Gamma_1\times\Gamma_2: (q,x,a,y,u,w)$ meaning the machine is
        in state $q$ with the head on $a$ everything to the right of head being $x$,
        everything to the left of
        the head being $y$, $u$ on the top of $\Gamma_1$ and $w$ on the top of $\Gamma_2$
      \end{subquestion}
      \begin{subquestion}{yields in one step:}
        $(q_1, x_1\underline{a_1}y_1, u\gamma, w\gamma)\vdash(q_2, x_2\underline{a_2}y_2,
        \Gamma_1*\gamma, \Gamma_2*\gamma)$ where the difference between the $x,a,y$ are only
        their positions on the string.
      \end{subquestion}
      \begin{subquestion}{language accepted:}
        $M$ accepts $L$ iff $\forall w\in L, (s, \#w\#, \gamma, \gamma)\vdash^*(h,\#w\#,
        \Gamma_1*, \Gamma_2*)$ basically all the strings must halt.
      \end{subquestion}
    \end{subquestion}

    \begin{subquestion}{b)}
      its the same as a turing maching but i
      dont know how to prove that formally.
    \end{subquestion}
  \end{question}


\end{document}
