\documentclass[11pt]{article}
\usepackage{mathtools}
\usepackage{mdframed}
\usepackage{fullpage}
\usepackage{amsfonts}
\usepackage{tikz}
\usepackage{fancyhdr}
\usepackage{lastpage}


%edit this for each class
\newcommand\name{John Collin Vincent}
\newcommand\classname{Com S 331}
\newcommand\assignment{Homework 11}


\newcounter{excounter}
\setcounter{excounter}{1}
\newcommand\ques[2]{\vskip 1em  \noindent\textbf{\arabic{excounter}\addtocounter{excounter}{1}.} \emph{#1} \noindent#2}
\newenvironment{question}{\ques{}\begin{quote}}{\end{quote}}
\newenvironment{subquestion}[1]{#1) \begin{quote}}{\end{quote}}

\pagestyle{fancy}
\rfoot{\name, page \thepage/\pageref{LastPage}}
\cfoot{}
\rhead{}
\lhead{}
\renewcommand{\headrulewidth}{0pt}
\renewcommand{\footrulewidth}{0pt}
\DeclarePairedDelimiter\ceil{\lceil}{\rceil}
\DeclarePairedDelimiter\floor{\lfloor}{\rfloor}


\begin{document}


  {\bf \classname \hspace{1cm} \assignment\hfill \name}
  \vskip 2em


  \begin{question}
    \begin{subquestion}{a}
      This Language is acceptable but not decidable because the only way to tell if the tape of $M$ is bounded on $w$ is to run $M$ on $w$ and check to
      see if it reaches a state where it can no longer access new tape spaces. This means it doesn't have to determain if $M$ halts on $w$ so it has a better
      chance of halting than $M$, since it only has to make sure that the possible transistions for $M$ no longer bring it to a new space in the tape. However if $M$
      does not halt and it is constantly in a state where it can access new tape spaces then the machine that accepts $L$ will not be able to halt, meaning
      it cannot be decidable.
    \end{subquestion}
    \begin{subquestion}{b}
      This Language is acceptable but not decidable because the only way to determain $n$ is to simulate the machine $M$ and then read the number of tape spaces
      used by $M$. Then it can check to make sure that the number of trailing ones is at most  $n$. If the machine $M$ hangs then this machine will also hang so
      it cannot always be decidable. It can however sometimes halt and return false.
    \end{subquestion}
  \end{question}

  \begin{question}
    \begin{subquestion}{a}
      this is not turing acceptable because with any given alphabet you can generate a infinite number of possible input strings mean the
      machine could never halt if you try to simulate $M$ for every possible input.
    \end{subquestion}
    \begin{subquestion}{b}
      this is turing acceptable but not decidable because you can simulate $M$ for every possible input string and run them in a dovetail mannor then as soon as
      at least 10 of the strings are accepted by $M$ the machine would halt with $\rho(M)$ being in the language. If $M$ hangs than it also hangs
    \end{subquestion}
    \begin{subquestion}{c}
      This is Turing decidable because you could simulate the first 10 steps of $M$ and if it accepts $w$ then its accepted otherwise its rejected.
    \end{subquestion}
    \begin{subquestion}{d}
      This is turing acceptable because it can simulate $M$ on $w$ and if $M$ accepts $w$ in less than 10 steps it rejects, but if $M$ halts at 10 or more steps
      than it accepts, but if $M$ hangs than it hangs as well.
    \end{subquestion}
  \end{question}

  \begin{question}
    If a machine $M$ exist that decides this language than we could use this machine to decide the language $\{\rho(M_1): L(M_1) = \Sigma*\}$ the second machine
    $M_2$ would then have to be equal to $\Sigma*$ since there is no larger set. This would mean $M$ would have to be able to decide a language that was proven to
    be undecidable in the Language book $L(M_1) = L(M_2)$\\
    \[ M\searrow\rho(M_1)\rho(M_2) \Leftrightarrow L(M_1) = L(M_2) \Leftrightarrow L(M) = \Sigma*\]
  \end{question}


\end{document}
