\documentclass[11pt]{article}
\usepackage{mathtools}
\usepackage{mdframed}
\usepackage{fullpage}
\usepackage{amsfonts}
\usepackage{tikz}
\usepackage{fancyhdr}
\usepackage{lastpage}


%edit this for each class
\newcommand\name{John Collin Vincent}
\newcommand\classname{Com S 331}
\newcommand\assignment{Homework 13}


\newcounter{excounter}
\setcounter{excounter}{1}
\newcommand\ques[2]{\vskip 1em  \noindent\textbf{\arabic{excounter}\addtocounter{excounter}{1}.} \emph{#1} \noindent#2}
\newenvironment{question}{\ques{}\begin{quote}}{\end{quote}}
\newenvironment{subquestion}[1]{#1) \begin{quote}}{\end{quote}}

\pagestyle{fancy}
\rfoot{\name, page \thepage/\pageref{LastPage}}
\cfoot{}
\rhead{}
\lhead{}
\renewcommand{\headrulewidth}{0pt}
\renewcommand{\footrulewidth}{0pt}
\DeclarePairedDelimiter\ceil{\lceil}{\rceil}
\DeclarePairedDelimiter\floor{\lfloor}{\rfloor}


\begin{document}


  {\bf \classname \hspace{1cm} \assignment\hfill \name}
  \vskip 2em


  \begin{question}
    \vspace{-1cm}
    \begin{align*}
      S &\rightarrow QE\\
      Q &\rightarrow 0QZ \hspace{.2cm}| \hspace{.2cm} 1QN\hspace{.2cm} | \hspace{.2cm}F\\
      FN &\rightarrow FR1\\
      FZ &\rightarrow FR0\\
      F &\rightarrow \epsilon\\
      R0N &\rightarrow NR0\\
      R0Z &\rightarrow ZR0\\
      R0E &\rightarrow 0\\
      R1N &\rightarrow NR1\\
      R1Z &\rightarrow ZR1\\
      R1E &\rightarrow 1\\
      R11 &\rightarrow 11\\
      R10 &\rightarrow 10\\
      R01 &\rightarrow 01\\
      R00 &\rightarrow 00
    \end{align*}
    basically it generates $ww^{\reflectbox{R}}$ which a normal grammer can do and then reverses $w^{\reflectbox{R}}$ with the power of an unrestricted grammar to form $ww$.
  \end{question}
  \vspace{-.2cm}
  \begin{question}
    \vspace{-.3cm}
    the assignment didn't ask for proofs so I'm hoping that these intuitions are enough of a answer.\\
    \begin{subquestion}{a}

      This language as acceptable because $M$ will use a finite number of cells when it halts or when it gets caught in a loop over the same section of tape which can be
      detected by tracking the machines configurations. In both of these cases we can accept $\rho(M)\rho(w)$, however for use to be able to decided that it will not
      use a finite number of cells we would have to be able to decide that $M$ will not halt on $w$ and if we could do that we could solve the halting problem.
    \end{subquestion}
    \begin{subquestion}{b}

      This can be decided because if $M$ halts using $n$ or less cells we can accept $\rho(M)\rho(w)01^n0$, and if we track the configurations of $M$ and detected that a configuration
      is repeated after using less than $n$ cells we know that it will continue to loop over those $n$ cells, allowing us to accept $\rho(M)\rho(w)$ even if $M$ doesn't halt. If
      however $M$ ever goes past $n$ cells we can halt and reject. Allowing us to decided $L$
    \end{subquestion}
  \end{question}

\end{document}
