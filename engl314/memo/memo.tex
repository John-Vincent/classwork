%%%%%%%%%%%%%%%%%%%%%%%%%%%%%%%%%%%%%%%%%
% Memo
% LaTeX Template
% Version 1.0 (30/12/13)
%
% This template has been downloaded from:
% http://www.LaTeXTemplates.com
%
% Original author:
% Rob Oakes (http://www.oak-tree.us) with modifications by:
% Vel (vel@latextemplates.com)
%
% License:
% CC BY-NC-SA 3.0 (http://creativecommons.org/licenses/by-nc-sa/3.0/)
%
%%%%%%%%%%%%%%%%%%%%%%%%%%%%%%%%%%%%%%%%%

\documentclass[letterpaper,11pt]{texMemo} % Set the paper size (letterpaper, a4paper, etc) and font size (10pt, 11pt or 12pt)

\usepackage{parskip} % Adds spacing between paragraphs
\setlength{\parindent}{15pt} % Indent paragraphs

%----------------------------------------------------------------------------------------
%	MEMO INFORMATION
%----------------------------------------------------------------------------------------

\memoto{Seniors preparing for a career in Software Engineering} % Recipient(s)

\memofrom{John Vincent} % Sender(s)

\memosubject{Communication through commentation} % Memo subject

\memodate{Tuesday, August 29, 2017} % Date, set to \today for automatically printing todays date

\logo{\includegraphics[width=0.3\textwidth]{logo.png}} % Institution logo at the top right of the memo, comment out this line for no logo

%----------------------------------------------------------------------------------------

\begin{document}

\maketitle % Print the memo header information

%----------------------------------------------------------------------------------------
%	MEMO CONTENT
%----------------------------------------------------------------------------------------

  \indent When working in a team it is critically important for engineers to write code that
  other people the team can easily understand and use. The most effective way to make sure
  that others can use code you create, regardless of its complexity, is to properly comment it.
  When it comes to comments uniformity can allow for users to more quickly find the information
  they are looking for. Now that the importance of using documentation is clear, its important
  to understand how to do the documentation properly.

  \indent The proper comment style to use can depend on a few things. First many large
  companies, like google, will issue a style guide(see link 1). These documents specify
  how everything should be formatted, and can include sections on how the company wants
  their code commented. Using the company style is always the best option because people
  in the company will be accustomed to reading this type of comment and will allow co workers
  to more quickly read, and understand your code. If there is no defined style guide, many
  languages have a standard comment style defined by its community.

  \indent The information a user needs in order to make use of some code largely depends
  on the langauge being used. So when trying to find a style to use it's always useful
  to look a the community behind the language. For example most java programmers use
  a style of comment called javadoc(see link 2) because it more clearly identifies the important
  aspects of the code that another user must interact with. People in the Perl community
  on the other hand would rather use POD(see link 3) commenting because it caters to perls more
  loose nature. An added benefit of using these language specific documentation styles is that
  many of them have tools that will generate a set of static webpages for the project, allowing
  users to have a clean, abstracted interface to reference when using the code. An example of
  a website generated using javadoc would be Apache's dbcp(see link 4) which is an all generated by
  the javadoc program.

  \begin{center}
    Links
  \end{center}
  \begin{tabular}{ l l l }
    1 & google style guide & google.github.io/styleguide/cppguide.html\#Comments\\
    2 & javadoc & www.tutorialspoint.com/java/java\_documentation.htm\\
    3 & POD & perldoc.perl.org/perlpod.html\#Command-Paragraph \\
    4 & dbcp doc & commons.apache.org/proper/commons-dbcp/api-1.2.2/index.html
  \end{tabular}

%----------------------------------------------------------------------------------------

\end{document}
