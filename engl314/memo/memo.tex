%%%%%%%%%%%%%%%%%%%%%%%%%%%%%%%%%%%%%%%%%
% Memo
% LaTeX Template
% Version 1.0 (30/12/13)
%
% This template has been downloaded from:
% http://www.LaTeXTemplates.com
%
% Original author:
% Rob Oakes (http://www.oak-tree.us) with modifications by:
% Vel (vel@latextemplates.com)
%
% License:
% CC BY-NC-SA 3.0 (http://creativecommons.org/licenses/by-nc-sa/3.0/)
%
%%%%%%%%%%%%%%%%%%%%%%%%%%%%%%%%%%%%%%%%%

\documentclass[letterpaper,11pt]{texMemo} % Set the paper size (letterpaper, a4paper, etc) and font size (10pt, 11pt or 12pt)
\nonstop
\usepackage{parskip} % Adds spacing between paragraphs
\setlength{\parindent}{15pt} % Indent paragraphs

%----------------------------------------------------------------------------------------
%	MEMO INFORMATION
%----------------------------------------------------------------------------------------

\memoto{Seniors preparing for a career in Software Engineering} % Recipient(s)

\memofrom{John Vincent} % Sender(s)

\memosubject{Communication through commentation} % Memo subject

\memodate{Thursday, August 31, 2017} % Date, set to \today for automatically printing todays date

\logo{\includegraphics[width=0.3\textwidth]{logo.png}} % Institution logo at the top right of the memo, comment out this line for no logo

%----------------------------------------------------------------------------------------

\begin{document}

\maketitle % Print the memo header information

%----------------------------------------------------------------------------------------
%	MEMO CONTENT
%----------------------------------------------------------------------------------------

\indent When working in a team it is critically important for engineers to write code that
other people the team can easily understand and use. The most effective way to make sure
that others can use code you create, regardless of its complexity, is to properly comment it.
When it comes to comments uniformity can allow for users to more quickly find the information
they are looking for. The common comment practice depends on the language being used,
Java uses JavaDOC, and Perl uses POD for example.

%----------------------------------------------------------------------------------------

\end{document}
