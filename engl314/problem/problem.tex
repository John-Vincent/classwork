\documentclass[11pt]{article}
\usepackage{mathtools}
\usepackage{mdframed}
\usepackage{fullpage}
\usepackage{amsfonts}
\usepackage{tikz}
\usepackage{hyperref}



%edit this for each class
\newcommand\name{John Vincent}
\newcommand\classname{}
\newcommand\assignment{}



\newcounter{excounter}
\setcounter{excounter}{1}
\newcommand\question[2]{\vskip 1em  \noindent\textbf{\arabic{excounter}\addtocounter{excounter}{1}.} \emph{#1} \noindent#2}


% You can also erase this if you do not have package fancyhdr
% Fancy footnote.........
\usepackage{fancyhdr}  %% If it does not work with your latex installation, you may just delete this...
\pagestyle{fancy}
\usepackage{lastpage}
\rfoot{\name, page \thepage/\pageref{LastPage}}
\cfoot{}
\rhead{}
\lhead{}
\renewcommand{\headrulewidth}{0pt}
\renewcommand{\footrulewidth}{0pt}
\DeclarePairedDelimiter\ceil{\lceil}{\rceil}
\DeclarePairedDelimiter\floor{\lfloor}{\rfloor}



\begin{document}


  \hfill \name\\
  \null\hfill Topic Proposal\\
  \vskip 1em

  \noindent\large{\textbf{Problem Statement}}

  \noindent On August 25, 1991 a man named Linus Torvalds posted a message announcing his new creation. It was a free kernel that was developed to mimic the behavior of the very popular UNIX terminal.
  Since he released the source code for others to use, they were able to help build features into this new creation(Hayward 2012, p.1). In only a few years different branches of development centered around this
  kernel started to sprout up, and iterations continue to be developed to this day, the most famous being the Android platform which runs 38\% of computers in the world(Operating System Market 2017).\\

  \noindent Due to the open nature of development this operating system is easy for knowledgeable people to take apart and rebuild to better fit the needs of a user.
  This flexibility, efficiency, and lack of licensing fees has caused rapid adoption among companies, and developers. Servers that run Linux are on the rise at the expense of UNIX, Windows, and mainframe
  technologies, and companies have an increasing need for developers train the using Linux (Vaughan-Nichols 2015). Aside from being a marketable skill the Linux environment has many tools made for free
  by developers for developers that increase productivity, and reduce frustration. This would explain why Linux is the second most used platform and the most loved platform among professional
  developers(Stack Overflow Developer Survey 2017).\\

  \noindent Despite all these reasons to incentivise students to learn Linux there are still very few times Linux is even mentioned to Software
  Engineering students at Iowa State. The first class on the flow chart that recommends Linux is Com S 327, and it isn't even a required course since embeded systems is an alternative.
  This class also comes later in the program after most students have already become attached to a development environment, causing a reluctance to adopt a new system. This course
  also recommends a VM approach which will ultimately end up in most students only learning the bare minimum of Linux and then tossing it to the side when the course is over since they
  never actuall have the operating system installed. On top of that even if they did like using Linux the course offers no resources on how to actually install it to their system
  if they did want to make the change. So this course does give students a reason to try using Linux, but it leaves them to their own devices to learn how to set up, and
  maintain a linux installation.\\

  \noindent There is a course offered by Iowa State, Com S 252, about "Introduction to installation, utilization, and administration of Linux systems". Which would be the perfect course
  for students to learn about Linux. This course, however, isn't on the radar for most students in Software Engineering. The marketability of understanding Linux, and the productivity benefits in using Linux for development
  make this course a must for future students. This course should be added to the Software Engineering graduation requirements which will ensure that graduates from Iowa State will be able to function
  in a work environment that uses linux servers, or develops for android, which is an inevitability for a career developer.

  \noindent\large{\textbf{Solution}}

  \noindent There are a number of ways to painlessly increase Linux literacy among students


  \clearpage

  \begin{center} \large{\textbf{References}} \end{center}

  \noindent Vaughan-Nichols, Steven J. “Linux Foundation Finds Enterprise Linux Growing at Windows' Expense.” ZDNet, ZDNet, 4 Dec. 2015,\\ www.zdnet.com/article/Linux-foundation-finds-enterprise-Linux-growing-at-windows-expense/.\\

  \noindent “Stack Overflow Developer Survey 2017.” Stack Overflow,\\ insights.stackoverflow.com/survey/2017\#most-popular-technologies.\\

  \noindent “The Complete Beginner's Guide to Linux.” Linux.com | The Source for Linux Information, 13 Aug. 2014, www.Linux.com/learn/complete-beginners-guide-Linux\%20.\\

  \noindent Hayward, David. “The History of Linux: How Time Has Shaped the Penguin.”TechRadar, TechRadar pro IT Insights for Business, 22 Nov. 2012,\\ www.techradar.com/news/software/operating-systems/the-history-of-Linux-how-time-has-shaped-the-penguin-1113914.\\

  \noindent “Operating System Market Share Worldwide.” StatCounter Global Stats,\\ gs.statcounter.com/os-market-share\#monthly-201704-201704-bar.

\end{document}
