\documentclass[11pt]{article}
\usepackage{mathtools}
\usepackage{mdframed}
\usepackage{fullpage}
\usepackage{amsfonts}
\usepackage{tikz}
\usepackage{fancyhdr}
\usepackage{lastpage}
\usepackage{listings}
\usepackage{xcolor}
\lstset { %
    language=C++,
    backgroundcolor=\color{black!5}, % set backgroundcolor
    basicstyle=\footnotesize,% basic font setting
}


%edit this for each class
\newcommand\name{John Collin Vincent}
\newcommand\classname{Com S 352}
\newcommand\assignment{Homework 5}


\newcounter{excounter}
\setcounter{excounter}{1}
\newcommand\ques[2]{\vskip 1em  \noindent\textbf{\arabic{excounter}\addtocounter{excounter}{1}.} \emph{#1} \noindent#2}
\newenvironment{question}{\ques{}\begin{quote}}{\end{quote}}
\newenvironment{subquestion}[1]{#1) \begin{quote}}{\end{quote}}

\pagestyle{fancy}
\rfoot{\name, page \thepage/\pageref{LastPage}}
\cfoot{}
\rhead{}
\lhead{}
\renewcommand{\headrulewidth}{0pt}
\renewcommand{\footrulewidth}{0pt}
\DeclarePairedDelimiter\ceil{\lceil}{\rceil}
\DeclarePairedDelimiter\floor{\lfloor}{\rfloor}


\begin{document}


  {\bf \classname \hspace{1cm} \assignment\hfill \name}
  \vskip 2em


  \begin{subquestion}{5.10}
    if a user program can disable interrupts then it can prevent the timer from pausing execution to let another process run potentially causing the
    system to hang while this process operates since important system control processes will not be able to regain control of the cpu.
  \end{subquestion}

  \begin{subquestion}{5.11}
    This will only disable interrupts on the one cpu, so other processors will not be affected by this change. This would allow other cores to run processes that
    could access the critical state.
  \end{subquestion}

  \begin{subquestion}{5.16}
    there needs to be a queue with each semaphore and the processes trying to access the critical section should be blocked and placed into the queue when there is another
    process already in the critical section. the wakup should then pop the first process off the queue and allow it to start operating.
  \end{subquestion}

  \begin{subquestion}{5.23}
    \begin{lstlisting}
      int l;

      void wait(semaphore *s){
        while(test_and_set(&l));
        s->val--;
        if(s->val < 0){
          s->list->enqueue(getpid());
          l = 0;
          block();
        } else
          l = 0;
      }

      void signal(semaphore *s){
        while(test_and_set(&l));
        s->val++;
        if(s->val <= 0){
          wake(s->list->pop());
        }
        l = 0;
      }
    \end{lstlisting}
  \end{subquestion}

  \begin{subquestion}{5.35}
    \begin{verbatim}
      sleeper: monitor
      begin
        processes: integer
        ticking: boolean
        StartTick, EndTick: condition;

        procedure sleep(x: integer);
        begin
          while(x > 0)
            if ticking then EndTick.wait;
            processes := processes + 1;
            EndTick.signal;
            StartTick.wait;
            x := x -1;
        end sleep;

        procedure tick();
        begin
          if ticking then EndTick.wait;
            ticking := true;
          while(processes <> 0)
            StartTick.signal;
            processes := processes - 1;
          ticking := false;
          EndTick.signal;
        end tick;
      end sleeper;

    \end{verbatim}
  \end{subquestion}


\end{document}
