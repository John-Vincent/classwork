\documentclass[11pt]{article}
\usepackage{mathtools}
\usepackage{mdframed}
\usepackage{fullpage}
\usepackage{amsfonts}
\usepackage{tikz}
\usepackage{fancyhdr}
\usepackage{lastpage}
\usepackage{enumitem}


%edit this for each class
\newcommand\name{John Collin Vincent}
\newcommand\classname{coms 352}
\newcommand\assignment{assignment 7}


\newcounter{excounter}
\setcounter{excounter}{1}
\newcommand\ques[2]{\vskip 1em  \noindent\textbf{\arabic{excounter}\addtocounter{excounter}{1}.} \emph{#1} \noindent#2}
\newenvironment{question}{\ques{}\begin{quote}}{\end{quote}}
\newenvironment{subquestion}[1]{#1) \begin{quote}}{\end{quote}}

\pagestyle{fancy}
\rfoot{\name, page \thepage/\pageref{LastPage}}
\cfoot{}
\rhead{}
\lhead{}
\renewcommand{\headrulewidth}{0pt}
\renewcommand{\footrulewidth}{0pt}
\DeclarePairedDelimiter\ceil{\lceil}{\rceil}
\DeclarePairedDelimiter\floor{\lfloor}{\rfloor}


\begin{document}


  {\bf \classname \hspace{1cm} \assignment\hfill \name}
  \vskip 2em


  \begin{subquestion}{8.13}
    \begin{subquestion}{contiguous}
      \begin{enumerate}[label=(\alph*)]
        \item External fragmentation is a real problem for contiguous memory organization because
          there can end up being many small holes in main memory but not a single hole large enough for
          the current requested allocation. This then requires memory compaction.
        \item Internal fragmentation isn't a problem with continuous allocation because the exact amount of
          memory request is allocated in main memory
        \item since memory is protected by giving a limit to the about of address that are accessible after the index
          of the allocated section it is not possible to share portions of memory between processes
      \end{enumerate}
    \end{subquestion}
    \begin{subquestion}{segmentation}
      \begin{enumerate}[label=(\alph*)]
        \item External fragmentation can still occurs with segmentation but its less likely than contiguous since
          the memory can be broken up into separate small segments allowing a single allocation to be fulfilled
          by multiple small holes of free memory.
        \item Internal fragmentation isn't a problem with pure segmentation because the exact amount of
          memory request is allocated in main memory potentially in multiple segments
        \item different processes can share a segments of their memory with each other allowing shared
          memory between processes.
      \end{enumerate}
    \end{subquestion}
    \begin{subquestion}{paging}
      \begin{enumerate}[label=(\alph*)]
        \item external fragmentation is not possible with paging since the memory is split into a specific number
          of pages.
        \item Internal fragmentation is a problem with paging because if you have a page size of 1024 and
          something needs to allocate 800 you have to allocate the whole 1024 causing 224 unused bytes.
        \item different processes can share pages of memory with each other allowing shared
          memory between processes.
      \end{enumerate}
    \end{subquestion}
  \end{subquestion}

  \begin{subquestion}{8.28}
    \begin{enumerate}[label=(\alph*)]
      \item 649
      \item 2310 (this is out of the segment)
      \item 590 (this is out of the segment)
      \item 1727
      \item 2064 (out of the segment)
    \end{enumerate}
  \end{subquestion}

  \begin{subquestion}{8.29}
    this allows multilevel paging like the 4 level paging found in Linux.
  \end{subquestion}

  \begin{subquestion}{9.3}
    \begin{enumerate}
      \item 0x9EF = 0x0EF
      \item 0x111 = 0x211
      \item 0x700 = 0xD00
      \item 0x0FF = 0xEFF
    \end{enumerate}
  \end{subquestion}

  \begin{subquestion}{9.8}
    \begin{subquestion}{1 frame}
      20 page faults for all
    \end{subquestion}
    \begin{subquestion}{2 frames}
      18 page faults for LRU
      18 page faults for FIFO
      15 page faults for optimal
    \end{subquestion}
    \begin{subquestion}{3 frames}
      15 page faults for LRU
      16 page faults for FIFO
      12 page faults for optimal
    \end{subquestion}
    \begin{subquestion}{4 frames}
      10 page faults for LRU
      14 page faults for FIFO
      10 page faults for optimal
    \end{subquestion}
    \begin{subquestion}{5 frames}
      8 page faults for LRU
      10 page faults for FIFO
      8 page faults for optimal
    \end{subquestion}
    \begin{subquestion}{6 frames}
      7 page faults for LRU
      10 page faults for FIFO
      7 page faults for optimal
    \end{subquestion}
    \begin{subquestion}{7 frames}
      since there are 7 frames and only 7 unique pages referenced all of the algorithms will behave the same
      since they will just fill the empty frames with the pages as they are referenced and there will
      never bee a need to replace a page in a frame. there will be 7 page faults for all of them as they
      fill them empty frames with the pages.
    \end{subquestion}
  \end{subquestion}
\end{document}
