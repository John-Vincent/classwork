\documentclass[11pt]{article}
\usepackage{mathtools}
\usepackage{mdframed}
\usepackage{fullpage}
\usepackage{amsfonts}
\usepackage{tikz}
\usepackage{fancyhdr}
\usepackage{lastpage}


%edit this for each class
\newcommand\name{John Collin Vincent}
\newcommand\classname{}
\newcommand\assignment{}


\newcounter{excounter}
\setcounter{excounter}{1}
\newcommand\ques[2]{\vskip 1em  \noindent\textbf{\arabic{excounter}\addtocounter{excounter}{1}.} \emph{#1} \noindent#2}
\newenvironment{question}{\ques{}\begin{quote}}{\end{quote}}
\newenvironment{subquestion}[1]{#1) \begin{quote}}{\end{quote}}

\pagestyle{fancy}
\rfoot{\name, page \thepage/\pageref{LastPage}}
\cfoot{}
\rhead{}
\lhead{}
\renewcommand{\headrulewidth}{0pt}
\renewcommand{\footrulewidth}{0pt}
\DeclarePairedDelimiter\ceil{\lceil}{\rceil}
\DeclarePairedDelimiter\floor{\lfloor}{\rfloor}


\begin{document}


  {\bf \classname \hspace{1cm} \assignment\hfill \name}
  \vskip 2em


  \begin{subquestion}{7.12}
    The locks cannot be shared when allocated for writing so there is mutual exclusion there.
    multiple locks could be requested by one thread causing a hold and wait.
    If a thread has a write lock another thread cannot take the lock away preemptively
    You could have the threads programmed in a way to circularly require the locks for writing.
    this means that all 4 necessary and sufficient conditions are met for deadlock in this system.
    For example you could have two threads A and B and two read-write locks $\alpha$ and $\beta$.
    A could allocate $\alpha$ for writing while B allocates $\beta$ for writing, then before
    either of them release their locks A could request $\beta$ for reading or writing and B could
    request $\alpha$ for reading or writing causing a deadlock in the system.
  \end{subquestion}


\end{document}
