\documentclass[11pt]{article}
\usepackage{mathtools}
\usepackage{mdframed}
\usepackage{fullpage}
\usepackage{amsfonts}
\usepackage{tikz}
\usepackage{fancyhdr}
\usepackage{lastpage}


%edit this for each class
\newcommand\name{John Collin Vincent}
\newcommand\classname{Com S 352}
\newcommand\assignment{Homework 6}


\newcounter{excounter}
\setcounter{excounter}{1}
\newcommand\ques[2]{\vskip 1em  \noindent\textbf{\arabic{excounter}\addtocounter{excounter}{1}.} \emph{#1} \noindent#2}
\newenvironment{question}{\ques{}\begin{quote}}{\end{quote}}
\newenvironment{subquestion}[1]{#1) \begin{quote}}{\end{quote}}

\pagestyle{fancy}
\rfoot{\name, page \thepage/\pageref{LastPage}}
\cfoot{}
\rhead{}
\lhead{}
\renewcommand{\headrulewidth}{0pt}
\renewcommand{\footrulewidth}{0pt}
\DeclarePairedDelimiter\ceil{\lceil}{\rceil}
\DeclarePairedDelimiter\floor{\lfloor}{\rfloor}


\begin{document}


  {\bf \classname \hspace{1cm} \assignment\hfill \name}
  \vskip 2em


  \begin{subquestion}{7.12}
    The locks cannot be shared when allocated for writing so there is mutual exclusion there.
    multiple locks could be requested by one thread causing a hold and wait.
    If a thread has a write lock another thread cannot take the lock away preemptively
    You could have the threads programmed in a way to circularly require the locks for writing.
    this means that all 4 necessary and sufficient conditions are met for deadlock in this system.
    For example you could have two threads A and B and two read-write locks $\alpha$ and $\beta$.
    A could allocate $\alpha$ for writing while B allocates $\beta$ for writing, then before
    either of them release their locks A could request $\beta$ for reading or writing and B could
    request $\alpha$ for reading or writing causing a deadlock in the system.
  \end{subquestion}

  \begin{subquestion}{7.17}
    Since the 3 processes need at most 2 resources the max number of edges is 6 while the number of nodes in the graph is 7(3 processes + 4 resources).
    Assuming that the any resource can be assigned to a process we know that a tree can be formed from the resource allocation graph since trees have
    $n$ vertices and $n-1$ edges. Trees must be acyclic so the graph will be acyclic. Also a connected graph with $n$ vertices that contains a cycle must have at least
    $n$ edges so we know this graph cannot contain a cycle. if the resouce allocation graph is acyclic than it cannot have deadlocks.
  \end{subquestion}

  \clearpage

  \begin{subquestion}{7.23}
    \begin{subquestion}{a}
      \begin{center}
        \vspace{-.2cm}
        \underline{Need}\\
        \vspace{.2cm}
        \begin{tabular}{l | c | c | c | c|}
                & A & B & C & D\\\hline
          $P_0$ & 2 & 2 & 1 & 1\\
          $P_1$ & 2 & 1 & 3 & 1\\
          $P_2$ & 0 & 2 & 1 & 3\\
          $P_3$ & 0 & 1 & 1 & 2\\
          $P_4$ & 2 & 2 & 3 & 3
        \end{tabular}
      \end{center}
      order = \textless$P_0$, $P_3$, $P_4$, $P_1$, $P_2$\textgreater
    \end{subquestion}
    \begin{subquestion}{b}
      \begin{table}[h]
        \hspace{.8cm}\underline{Allocation} \hfill \underline{Need} \hfill \underline{Available}\hspace{1cm}\null\\
        \vspace{.2cm}
        \begin{tabular}[t]{l | c | c | c | c|}
                & A & B & C & D\\\hline
          $P_0$ & 2 & 0 & 0 & 1\\
          $P_1$ & 4 & 2 & 2 & 1\\
          $P_2$ & 2 & 1 & 0 & 3\\
          $P_3$ & 1 & 3 & 1 & 2\\
          $P_4$ & 1 & 4 & 3 & 2
        \end{tabular}
        \hfill
        \begin{tabular}[t]{l | c | c | c | c|}
                & A & B & C & D\\\hline
          $P_0$ & 2 & 2 & 1 & 1\\
          $P_1$ & 1 & 0 & 3 & 1\\
          $P_2$ & 0 & 2 & 1 & 3\\
          $P_3$ & 0 & 1 & 1 & 2\\
          $P_4$ & 2 & 2 & 3 & 3
        \end{tabular}
        \hfill
        \begin{tabular}[t]{l | c | c | c | c|}
                & A & B & C & D\\\hline
                & 2 & 2 & 2 & 1
        \end{tabular}
      \end{table}
      order = \textless$P_0$, $P_3$, $P_4$, $P_1$, $P_2$\textgreater, so yes it can be granted immediately
    \end{subquestion}
    \begin{subquestion}{c}
      \begin{table}[h]
        \hspace{.8cm}\underline{Allocation} \hfill \underline{Need} \hfill \underline{Available}\hspace{1cm}\null\\
        \vspace{.2cm}
        \begin{tabular}[t]{l | c | c | c | c|}
                & A & B & C & D\\\hline
          $P_0$ & 2 & 0 & 0 & 1\\
          $P_1$ & 3 & 1 & 2 & 1\\
          $P_2$ & 2 & 1 & 0 & 3\\
          $P_3$ & 1 & 3 & 1 & 2\\
          $P_4$ & 1 & 4 & 5 & 2
        \end{tabular}
        \hfill
        \begin{tabular}[t]{l | c | c | c | c|}
                & A & B & C & D\\\hline
          $P_0$ & 2 & 2 & 1 & 1\\
          $P_1$ & 2 & 1 & 3 & 1\\
          $P_2$ & 0 & 2 & 1 & 3\\
          $P_3$ & 0 & 1 & 1 & 2\\
          $P_4$ & 2 & 2 & 1 & 3
        \end{tabular}
        \hfill
        \begin{tabular}[t]{l | c | c | c | c|}
                & A & B & C & D\\\hline
                & 3 & 3 & 0 & 1
        \end{tabular}
      \end{table}
      order = none, all processes require at least 1 additional c resource and $P_4$'s requestion would leave the system with 0 available c resources to use.
      if this request is granted there will be a deadlock in the system.
    \end{subquestion}
  \end{subquestion}

  \begin{subquestion}{7.24}
    the assumption is that none of the processes will request more resources.
    it can be violated by a process requesting more resources.
  \end{subquestion}

  \clearpage
  \begin{subquestion}{7.25}
    \begin{verbatim}
      bridge: monitor
      begin
        int southbound, northbound;
        boolean inuse = false;
        condition Bridge, South, North;

        procedure travelSouth();
        begin
          southbound++;
          if northbound > 0 then South.wait;
          if inuse then Bridge.wait;
          inuse = true;
          <travel south>
          inuse = false;
          Bridge.signal;
          southbound--;
          if northbound > 0 then North.signal;
        end travelSouth;

        procedure travelNorth();
        begin
          northbound++;
          if southbound > 0 then North.wait;
          if inuse then Bridge.wait;
          inuse = true;
          <travel north>
          inuse = false;
          Bridge.signal;
          northbound--;
          if southbound > 0 then South.signal;
        end travelNorth;
      end bridge;
    \end{verbatim}
  \end{subquestion}

\end{document}
