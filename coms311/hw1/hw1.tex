\documentclass[11pt]{article}
\usepackage{mathtools}
\usepackage{mdframed}
\usepackage{fullpage}
\usepackage{amsfonts}
\usepackage{tikz}            % if you delete this, you will have trouble with the included picture. You can delete the picture.



%edit this for each class
\newcommand\name{John Vincent} %%%%%%%%%%%%%%  WRITE YOUR NAME HERE
\newcommand\classname{COMS 311 Algorithms}
\newcommand\assignment{HW 1}




\newcounter{excounter}
\setcounter{excounter}{1}
\newcommand\question[2]{\vskip 1em  \noindent\textbf{\arabic{excounter}\addtocounter{excounter}{1})} \emph{#1} \noindent#2}


% You can also erase this if you do not have package hancyhdr
% Fancy footnote.........
\usepackage{fancyhdr}  %% If it does not work with your latex installation, you may just delete this...
\pagestyle{fancy}
\usepackage{lastpage}
\rfoot{\name, page \thepage/\pageref{LastPage}}
\cfoot{}
\rhead{}
\lhead{}
\renewcommand{\headrulewidth}{0pt}
\renewcommand{\footrulewidth}{0pt}
\DeclarePairedDelimiter\ceil{\lceil}{\rceil}
\DeclarePairedDelimiter\floor{\lfloor}{\rfloor}



\begin{document}


  {\bf \classname \hspace{1cm} \assignment\hfill \name}
  \vskip 2em


  \question{}
  \begin{quote}
    a) induction on upper bound of summation
    \begin{quote}
      \textit{\textbf{Basis:}} when $n=1, \sum_{i=1}^{n}i^3 = 1, (\sum_{i=1}^{n}i)^2 = 1$\\
      \textit{\textbf{Inductive Hypothesis: }} $\forall n \ge 1, \sum_{i=1}^{n}i^3 = (\sum_{i=1}^{n}i)^2$\\
      \textit{\textbf{Inductive Step: }}
      \begin{align*}
        \sum_{i=1}^{n+1}i^3 &= \left(\sum_{i=1}^{n+1}i\right)^2\\
        (n+1)^3 + \sum_{i=1}^{n}i^3 &= (n+1 + \sum_{i=1}^{n}i)(n+1 + \sum_{i=1}^{n}i)\\
        (n+1)^3 + \left(\sum_{i=1}^{n}i\right)^2 &= (n+1)^2 + 2(n+1)\sum_{i=1}^{n}i + \left(\sum_{i=1}^{n}i\right)^2\\
        (n+1)^3 &= (n+1)^2 + 2(n+1)(\frac{n(n+1)}{2})\\
        (n+1)(n+1)^2 &= (n+1)^2 + n(n+1)^2\\
        n+1 &= 1+n
      \end{align*}
      proving $\sum_{i=1}^{n}i^3 = (\sum_{i=1}^{n}i)^2, \forall n\ge1$ through mathematical induction
    \end{quote}
    b) induction on the term $n$
    \begin{quote}
      \textit{\textbf{Basis: }} when $n = 4, 2^n = 16, n! = 24$\\
      \textit{\textbf{Inductive Hypothesis: }} $\forall n \ge 4, 2^n < n!$\\
      \textit{\textbf{Inductive Step: }}
      \begin{align*}
        2^{n+1} &< (n+1)!\\
        2 * 2^{n} &< n! * (n+1)\\
        2 * 2^{n} &< n! * (n+1)
      \end{align*}
      since $2^{n}$ is always less than $n!$ by the Inductive Hypothesis the equation only depends on the following
      \begin{align*}
        2 &\le (n+1)\\
        1 &\le n
      \end{align*}
      since $n$ was defined as being greater than equal to 4, $n$ will always be greater than $1$ so this proves the claim
      by induction
    \end{quote}
  \end{quote}
  \vspace{2cm}
  \question{}
  \begin{quote}
    a)
    \begin{quote}
      \textit{\textbf{Claim:}} For every Full b-tree $T, n(T) \ge h(T).$\\
      \textit{\textbf{Base Case:}} The b-tree with only one node has $n(T) = 1$ and $h(T) = 1$ so the claim holds.\\
      \textit{\textbf{Inductive Hypothesis:}} assume $X, Y$ are two B-Trees such that $n(X) \ge h(X), n(X) \ge n(Y), h(X) \ge h(Y)$\\
      \textit{\textbf{Inductive Step:}} by the definition of a full b-tree we can create a new full b-tree $Z$
      by adding a new node, and making $X$ the right child of this node and $Y$ the left child of this node.
      this would make $n(Z) = 1 + n(X) + n(Y), h(Z) = 1 + h(X)$, and since $n(x) \ge h(X), n(Y) \ge h(X)$ this would make
      $n(Z) > h(Z)$\\
      this proves the claim by structural induction.
    \end{quote}
    b)
    \begin{quote}
      \textit{\textbf{Claim:}} For every Full Binary Tree $T, i(T) \ge h(T) - 1$\\
      \textit{\textbf{Base Case:}} consider the B-Tree $T$ with one node $i(T) = 0, h(T) = 1, i(T) = h(T) -1$\\
      \textit{\textbf{Inductive Hypothesis:}} assume $X, Y$ are two B-Trees such that $i(X) \ge h(X)-1, i(Y) \ge h(Y)-1, h(X) \ge h(Y)$\\
      \textit{\textbf{Inductive Step:}} a new full b-tree $Z$ can be constructed by adding a new node and make $X$ the right
      child of this node, and $Y$ the left child of this node. $h(Z) = h(X) + 1, i(Z) = i(X) + i(Y) + 1$, since $i(X) \ge h(X)-1$
      $i(X) \ge H(X) -1$\\
      this proves the claim by structural induction
    \end{quote}
    c)
    \begin{quote}
      \textit{\textbf{Claim:}} For every Full Binary Tree $T, \mathbb{l}(T) = (n(T) + 1)/2$\\
      \textit{\textbf{Base Case:}} consider the B-Tree $T$ with one node $\mathbb{l}(T) = 1, \frac{n(T)+1}{2} = 1$\\
      \textit{\textbf{Inductive Hypothesis:}} assume $X, Y$ are two B-Trees such that $\ell(X) = (n(X) + 1)/2, \ell(Y) = (n(Y) + 1)/2,
       n(X) \ge n(Y)$\\
      \textit{\textbf{Inductive Step:}}
    \end{quote}
  \end{quote}

\end{document}
