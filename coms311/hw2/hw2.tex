\documentclass[11pt]{article}
\usepackage{mathtools}
\usepackage{mdframed}
\usepackage{fullpage}
\usepackage{amsfonts}
\usepackage{tikz}



%edit this for each class
\newcommand\name{John Vincent}
\newcommand\classname{Com S 311}
\newcommand\assignment{Homework 2}



\newcounter{excounter}
\setcounter{excounter}{1}
\newcommand\question[2]{\vskip 1em  \noindent\textbf{\arabic{excounter}\addtocounter{excounter}{1}.} \emph{#1} \noindent#2}


% You can also erase this if you do not have package fancyhdr
% Fancy footnote.........
\usepackage{fancyhdr}  %% If it does not work with your latex installation, you may just delete this...
\pagestyle{fancy}
\usepackage{lastpage}
\rfoot{\name, page \thepage/\pageref{LastPage}}
\cfoot{}
\rhead{}
\lhead{}
\renewcommand{\headrulewidth}{0pt}
\renewcommand{\footrulewidth}{0pt}
\DeclarePairedDelimiter\ceil{\lceil}{\rceil}
\DeclarePairedDelimiter\floor{\lfloor}{\rfloor}



\begin{document}


  {\bf \classname \hspace{1cm} \assignment\hfill \name}
  \vskip 2em


  \question{}
  \begin{quote}
    a)
    \begin{align*}
      \forall n\ge1: 5n^2 - 2n + 26 \le 29n^2\implies 5n^2 -2n + 26\in O(n^2)
    \end{align*}
    b)
    \[
      \forall n > a: \frac{a^n}{n!} < \frac{a}{1} * \frac{a}{1} \ldots \frac{a}{a} * 1 \ldots  1 * \frac{a}{n} = \frac{a^{a}}{n}\implies\\
      \lim_{n\to\infty} \frac{a^n}{n!} = 0\implies a^n \in O(n!)
    \]
    c)
    \[
      \forall n \ge 1: 2^{n+a} = 2^a * 2^n = C * 2^n \implies 2^{n+a} \in O(2^n)
    \]
    d)
    \[
      \forall a \ge 1: \log_{a}n = \frac{\log_{2}n}{\log_{2}a}\implies f(n)\in O(\log_{a}n)
    \]
    e)
    \begin{align*}
      2^n &\le n^{\log^2n} * C\\
      n &\le \log_2(n^{\log^2n}) + \log_2(C)\\
      n &\le \frac{\log_2(n^{\log^2n})}{\log_2n} *\log_2n + \log_2(C)\\
      n &\le \log^2n * \frac{\log_2n}{\log_210} * \log_210 + C'\\
      n &\not\le \log^3n * \log_210 + C'\implies 2^n \notin O(n^{\log^2n})
    \end{align*}
    f) assume there is some constant such that $2^{2^{n+1}} \le C * 2^{2^n}$ then
    \begin{align*}
      \log_{2} (\log_{2} (2^{2^{n+1}})) &\le \log_{2} (\log_{2} (C * 2^{2^n}))\\
      n+1 &\le \log_{2} (\log_{2} (C) + 2^n) \\
      n+1 &\le n + \log_{2}(\frac{\log_{2} C}{2^n} + 1)\\
      n+1 &\le n + \log_{2}(\frac{C'}{2^n} + 1)\\
      1 &\not\leq \log_{2}(\frac{C'}{2^n} + 1) \implies 2^{2^{n+1}} \notin O(2^{2^n})
    \end{align*}
  \end{quote}

  \question{}
  \begin{quote}
    a)
    \begin{align*}
      \sum_{i=0}^{n-1}i = \frac{(n-1)(n-1+1)}{2}\implies O(n^2)
    \end{align*}
    b)
    \begin{quote}
      the best case time complexity is $O(n)$, and i happens when the median of the array is the first element. the worst case time
      complexity is $O(n^2)$ and that happens when the last element in the array is the median since it will loop through the entire
      array n times.
    \end{quote}
  \end{quote}
  \clearpage

  \question{}
  \begin{quote}
    start = 0;\\
    end = n;\\
    i = end/2;\\
    flag = false;\\
    while(!flag)\{\\
      \null\hspace{.5cm} flag = A[i] == 1 \&\& A[i-1] == 0;\\
      \null\hspace{.5cm} if(!flag)\{\\
      \null\hspace{1cm} if(A[i]==0)\{\\
      \null\hspace{1.5cm} start = i;\\
      \null\hspace{1.5cm} i = start + (end-start)/2;\\
      \null\hspace{1cm} \} else\{\\
      \null\hspace{1.5cm} end = i;\\
      \null\hspace{1.5cm} i = start + (end-start)/2;\\
      \null\hspace{1cm} \}\\
      \null\hspace{.5cm} \}\\
    \}\\
    return i;\\\\
    This continually divides the remaining length in half, resulting in $\log_{2}n$ elements being checked. which makes $O(\log n)$ the
    worst case time complexity.
  \end{quote}

  \question{}
  \begin{quote}
    for i in [1,k]\\
    \null\hspace{.5cm} $O(i*n)$\\
    \[
      n*k + n*(k-1)+\ldots+n*2 + n = \sum_{i=1}^{k}ni = n\frac{k(k+1)}{2}\implies O(k^2n)
    \]
  \end{quote}

  \question{}
  \begin{quote}
    a) 314,606,891, and 817,504,243 are the numbers i chose the GCD ran in 2085 milliseconds and the fastGCD ran in less than 1 millisecond\\
    b) 5,915,587,277, and 5,463,458,053 are the numbers i chose. GCD ran in 36,177 milliseconds and the fastGCD ran in less than 1 millisecond\\
    c) \\
    \null\hspace{.5cm}Case 1) $b \ge \frac{a}{2}$ after an iteration $a = b, b = a\%b$ making b at most $\frac{b}{2}$\\
    \null\hspace{.5cm}Case 2) $b \le \frac{a}{2}$ after an iteration $a = b, b = a\%b$ making a at most $\frac{a}{2}$\\
    \null\hspace{.5cm}since both cases end it a 50\% reduction the algorithm is $O(log_2max(a,b))$ but since $log_2a, log_2b = \text{number of bits in a and b}$
    the algorithm is $O(n)$ where n is the number of bits in a and b.
  \end{quote}

\end{document}
