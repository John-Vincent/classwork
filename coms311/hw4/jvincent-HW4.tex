\documentclass[11pt]{article}
\usepackage{mathtools}
\usepackage{mdframed}
\usepackage{fullpage}
\usepackage{amsfonts}
\usepackage{tikz}
\usepackage{fancyhdr}
\usepackage{lastpage}


%edit this for each class
\newcommand\name{John Collin Vincent}
\newcommand\classname{Com S 311}
\newcommand\assignment{Homework 4}


\newcounter{excounter}
\setcounter{excounter}{1}
\newcommand\ques[2]{\vskip 1em  \noindent\textbf{\arabic{excounter}\addtocounter{excounter}{1}.} \emph{#1} \noindent#2}
\newenvironment{question}{\ques{}\begin{quote}}{\end{quote}}


\pagestyle{fancy}
\rfoot{\name, page \thepage/\pageref{LastPage}}
\cfoot{}
\rhead{}
\lhead{}
\renewcommand{\headrulewidth}{0pt}
\renewcommand{\footrulewidth}{0pt}
\newcommand\tab{\null\hspace{.4cm}}
\DeclarePairedDelimiter\ceil{\lceil}{\rceil}
\DeclarePairedDelimiter\floor{\lfloor}{\rfloor}


\begin{document}


  {\bf \classname \hspace{1cm} \assignment\hfill \name}
  \vskip 2em


  \begin{question}
    a)\\
    \begin{center}
      \begin{tikzpicture}[level/.style={sibling distance=60mm/#1}]
        \node [circle,draw] (a){$59$}
          child {node [circle,draw] (b) {$12$}
            child {node [circle,draw] (c) {$9$}
              child {node [circle,draw] (d) {$2$}}
              child{node [circle,draw] (i) {$8$}}
            }
            child {node [circle,draw] (e) {$10$}}
          }
          child{node [circle,draw] (f) {$52$}
            child{node [circle,draw] (g) {$5$}}
            child{node [circle,draw] (h) {$13$}}
          };
      \end{tikzpicture}
    \end{center}
    b)
    \begin{quote}
      you would remove 59 and replace it with 8 then heapify down until 8 settles into a spot.
    \end{quote}
    \begin{center}
      \begin{tikzpicture}[level/.style={sibling distance=60mm/#1}]
        \node [circle,draw] (a){$52$}
          child {node [circle,draw] (b) {$12$}
            child {node [circle,draw] (c) {$9$}
              child {node [circle,draw] (d) {$2$}}
            }
            child {node [circle,draw] (e) {$10$}}
          }
          child{node [circle,draw] (f) {$13$}
            child{node [circle,draw] (g) {$5$}}
            child{node [circle,draw] (h) {$8$}}
          };
      \end{tikzpicture}
    \end{center}
  \end{question}

  \begin{question}
  \end{question}

  \begin{question}
    a)
    \begin{quote}
      \[ 2(n-2) + 1 = 2n-3 \]
    \end{quote}
    b)
    \begin{quote}
      B = Array[$\floor{log_2 n}$+1][n]\\
      B[0] = deepCopy(A)\\
      c = $\floor{n/2}$\\
      r = n\%2\\
      For i in range 1 to B.length\{\\
      \tab For j in range 0 to c\{\\
      \tab\tab B[i][j] = B[i-1][2j] $>$ B[i-1][2j+1] ? B[i-1][2j] : B[i-1][2j+1]\\
      \tab \}\\
      \tab if(r == 1)\{\\
      \tab\tab B[i][c] = B[i-1][2c+1]\\
      \tab \}\\
      \tab r = c\%2\\
      \tab c = $\floor{\text{c/2}}$\\
      \}%now just loop through and find the second largest elemnt, it must have been beaten by the largest somewhere
    \end{quote}
  \end{question}


\end{document}
